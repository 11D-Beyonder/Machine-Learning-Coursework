\documentclass{ctexart}
\usepackage{amsmath, amsfonts, amssymb} % 数学公式、符号
\usepackage[colorlinks,linkcolor=red,anchorcolor=blue,citecolor=green]{hyperref}
\usepackage[left=2.50cm, right=2.50cm, top=2.50cm, bottom=2.50cm]{geometry} %页边距
\usepackage{graphicx}   % 图片
\usepackage{multicol}
\usepackage{bm}
\usepackage{listings}
\usepackage{xcolor}
\usepackage{tikz}
\usetikzlibrary{arrows,shapes,chains}
\usepackage{multirow}
\usepackage{fancyhdr} %设置页眉、页脚
\pagestyle{fancy}
\lhead{}
\chead{}
\lfoot{}
\cfoot{}
\rfoot{}

% 用来设置代码的样式
\lstset{
	basicstyle          =   \sffamily,          % 基本代码风格
	keywordstyle        =   \bfseries,          % 关键字风格
	commentstyle        =   \rmfamily\itshape,  % 注释的风格,斜体
	stringstyle         =   \ttfamily,  % 字符串风格
	flexiblecolumns,                % 别问为什么,加上这个
	numbers             =   left,   % 行号的位置在左边
	showspaces          =   false,  % 是否显示空格,显示了有点乱,所以不现实了
	numberstyle         =   \zihao{-5}\ttfamily,    % 行号的样式,小五号,tt等宽字体
	showstringspaces    =   false,
	captionpos          =   t,      % 这段代码的名字所呈现的位置,t指的是top上面
	frame               =   lrtb,   % 显示边框
}

\lstdefinestyle{Python}{
	language        =   Python, % 语言选Python
	basicstyle      =   \zihao{-5}\ttfamily,
	numberstyle     =   \zihao{-5}\ttfamily,
	keywordstyle    =   \color{blue},
	keywordstyle    =   [2] \color{teal},
	stringstyle     =   \color{magenta},
	commentstyle    =   \color{red}\ttfamily,
	breaklines      =   true,   % 自动换行,建议不要写太长的行
	columns         =   fixed,  % 如果不加这一句,字间距就不固定,很丑,必须加
	basewidth       =   0.5em,
}

\title{\textbf{机器学习实验报告\\{\Large{决策树}}}} % 标题与子标题
\author{\sffamily{朱天泽}} % 作者
\date{(日期:\today)} % 日期

\begin{document}
	\maketitle\thispagestyle{fancy}
	% 摘要开始
	\noindent{\bf{摘要}}
	在《机器学习》第四章中,我学习了决策树模型。在此次实验中,我尝试用DFS和BFS实现了多个用于分类任务的决策树模型,包括针对离散值的单变量决策树、基于对数几率回归的多变量决策树。在训练好决策树模型的基础上,我使用了4个UCI数据集,比较了未剪枝、预剪枝、后剪枝三种类型决策树的准确率。此外,我尝试使用了机器学习库scikit-learn,同自己实现的算法进行了比较。
	
	\noindent{\bf{关键词}} 决策树;剪枝;分类;搜索;scikit-learn
	
	% 正文开始
	\section{习题1}
	\subsection{编程题目理解}
	题目要求:任意选择4个UCI数据集,对基于信息增益划分选择(ID3)、基于基尼指数划分选择(CART),基于对率回归划分选择的决策树算法(包括未剪枝、预剪枝、后剪枝三种)进行实验比较。
	
	ID3、CART两种划分选择,可用于实现单变量的决策树;对率回归是一个线性分类模型,基于对率回归,可以实现多变量决策树。
	\subsection{决策树模型原理阐述}
	
	\subsection{决策树模型设计思路}
	
	\subsection{核心代码详解} 
	\subsection{实验结果分析}

	\subsection{学习收获}
	
	\subsection{参考资料}   
	\begin{itemize}
		\item
	\end{itemize}

	\section{习题2}
	\subsection{编程题目理解}

	\subsection{交叉验证法阐述}
	
	\subsection{核心代码分析}

	\subsection{实验结果分析}
	
	\subsection{学习收获}
	
	\subsection{参考资料}
	
	\begin{itemize}
		\item
	\end{itemize}
	
	\section{习题3}
	\subsection{编程题目理解}
	\subsubsection{原理阐述}
	\subsection{设计思路}
	
	\subsection{核心代码详解}
	\subsubsection{数据集格式}
	
	\subsection{实验结果分析}

	\subsection{学习收获}

	
	\subsection{参考资料}
	\begin{itemize}
		\item
	\end{itemize}
\end{document}




%https://archive.ics.uci.edu/ml/datasets/Lymphography
%https://archive.ics.uci.edu/ml/datasets/balance+scale
%https://archive.ics.uci.edu/ml/datasets/Ionosphere
%https://archive.ics.uci.edu/ml/datasets/Tic-Tac-Toe+Endgame